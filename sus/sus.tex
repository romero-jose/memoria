\setlist[enumerate,1]{labelindent=0pt, leftmargin=*}
\pagestyle{empty}

\newcolumntype{P}{>{\centering\arraybackslash}p{0.75cm}}
\newcolumntype{L}{>{\raggedright\arraybackslash}m{0.2\textwidth}}
\newcolumntype{R}{>{\raggedleft\arraybackslash}m{0.2\textwidth}}

\newcommand{\usetbl}{%
  \begin{tabular}{@{}|*5{P|}@{}}
    \hline
    1 & 2 & 3 & 4 & 5 \\
    \hline
  \end{tabular}
}

\newcommand\prop[1]{%
  \item
  \parbox[t]{0.5\textwidth}{#1}%
  \hspace{1cm}
  \parbox[t]{0.5\textwidth}{\usetbl}%
}

\section*{Escala de Usabilidad de Sistemas (SUS)}

\vspace*{0.5cm}
Por favor seleccione de cada uno de los enunciados la opción que mejor describa su experiencia con la herramienta. Un puntaje de 1 significa que usted se encuentra totalmente en desacuerdo con el enunciado, mientras que un puntaje en 5 significa que está totalmente de acuerdo, un puntaje de 3 significaría que usted se encuentra neutral con el enunciado.

\vspace*{0.25cm}

\hspace*{0.5\textwidth}%
\hspace*{1cm}
\begin{tabularx}{0.5\textwidth}{@{}LR@{}}
\textbf{Totalmente} & \textbf{Totalmente} \\
\textbf{en desacuerdo} & \textbf{de acuerdo} \\
\end{tabularx}

\begin{enumerate}
\prop{Me gustaría usar esta herramienta frecuentemente}

\prop{Considero que esta herramienta es innecesariamente compleja}

\prop{Considero que la herramienta es fácil de usar}

\prop{Considero necesario el apoyo de personal experto para poder utilizar la herramienta}

\prop{Considero que las funciones de la herramienta están bien integradas}

\prop{Considero que la herramienta presenta muchas contradicciones}

\prop{Imagino que la mayoría de las personas aprenderían a usar esta herramienta rápidamente}

\prop{Considero que el uso de esta herramienta es tedioso}

\prop{Me sentí muy confiado al usar la herramienta}

\prop{Necesité saber bastantes cosas antes de poder empezar a usar la herramienta}

\end{enumerate}

\newpage

\section*{Preguntas abiertas}

\vspace*{1cm}

\textbf{¿Qué te gustó de la herramienta? ¿Por qué?}

\vspace*{3cm}

\textbf{¿Qué no te gustó de la herramienta? ¿Por qué?}

\vspace*{3cm}

\textbf{¿Cómo podría ser mejor la herramienta?}

\vspace*{3cm}

\newpage
\section*{Caracterización}

\begin{enumerate}

\item{Indica tu edad}
    \begin{itemize}
        \item[\framebox(10,10){ }]{18--19}
        \item[\framebox(10,10){ }]{20--21}
        \item[\framebox(10,10){ }]{22--23}
        \item[\framebox(10,10){ }]{24--25}
        \item[\framebox(10,10){ }]{26--27}
        \item[\framebox(10,10){ }]{28--29}
        \item[\framebox(10,10){ }]{30 o más}
    \end{itemize}

\item{Indica tu género}
    \begin{itemize}
        \item[\framebox(10,10){ }]{Masculino}
        \item[\framebox(10,10){ }]{Femenino}
        \item[\framebox(10,10){ }]{Otro, }
        \item[\framebox(10,10){ }]{Prefiero no responder}
    \end{itemize}

\item{Indica tu avance curricular. Responde según el mayor número asociado a los cursos que hayas cursado o estés cursando de la siguiente lista}
    \begin{itemize}
        \item[\framebox(10,10){ }] Introducción a la programación
        \item[\framebox(10,10){ }] Algoritmos y estructuras de Datos
        \item[\framebox(10,10){ }] Diseño y análisis de algoritmos
        \item[\framebox(10,10){ }] Lenguajes de programación
        \item[\framebox(10,10){ }] Proyecto de software
    \end{itemize}
\end{enumerate}
