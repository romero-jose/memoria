\chapter{Validación}

Para evaluar la solución se hicieron pruebas con usuarios y se les pidió que contestaran un cuestionario. 

Los participantes fueron estudiantes de la Facultad de Ciencias Físicas y Matemáticas que accedieron a ser parte de la investigación. Estos fueron reclutados usando el foro institucional y el grupo de Telegram de los estudiantes del Departamento de Ciencias de la Computación.

Las pruebas con usuarios consistieron en que a cada participante se le explicó el proceso, luego se le dió una breve explicación sobre cómo usar la herramienta y se le pidió que utilice la herramienta para generar una visualización de cierta operación. Posteriormente se le pidió que conteste un formulario.

El formulario tiene dos partes. La primera parte consiste en el System Usability Scale (SUS), mientras que la segunda parte está compuesta por preguntas abiertas sobre la herramienta.

El cuestionario SUS fue introducido en~\cite{brooke1996quick}.
