\chapter{Conclusión}

% Breve resumen del trabajo realizado

Se desarrolló \textit{dsvisualizer}, una herramienta para generar visualizaciones animadas de estructuras de datos para ser utilizada en Jupyter Notebooks. Esta herramienta permite a los usuarios visualizar sus propias implementaciones, agregando la menor cantidad de instrumentación necesaria a su código. Además, se evaluó esta herramienta realizando pruebas con 12 estudiantes de la Facultad de Ciencias Físicas y Matemáticas. El resultado de esta evaluación fue muy positivo, obteniendo un puntaje promedio de 90.0, el cual se encuentra en el percentil 99.8.

Se cumplió el objetivo general de ``\textit{Diseñar e implementar una librería para generar visualizaciones de estructuras de datos en Notebooks de Python que sea efectiva, eficiente y usable}'' y también se cumplieron todos los objetivos específicos.

Esta herramienta podrá ser utilizada cómo una ayuda para la docencia en el curso Algoritmos y Estructuras de Datos de la Facultad de Ciencias Física y Matemáticas, que beneficiará a los futuros estudiantes de esta facultad. Además, se encuentra disponible públicamente, por lo que cualquier persona que le interese podrá utilizarla para ayudarla en su entendimiento de las estructuras de datos. La herramienta cuenta con licencia BSD (Berkeley Software Distribution), lo que permite redistribuirla, usarla y modificarla con mínimas restricciones.

Durante el desarrollo de la memoria se aprendió mucho sobre distintos temas. Se aprendió sobre la generación de visualizaciones, primero utilizando Threejs y luego utilizando D3js al descubrir que Threejs no era la herramienta apropiada para este problema. Se aprendió sobre distintos mecanismos del lenguaje Python, la primera versión de la librería utilizaba herencia, luego se pasó por una versión que utilizaba metaclases y finalmente, se optó por utilizar decoradores, ya que esto permite una interfaz más cómoda para el usuario. Se aprendió sobre metodologías para la evaluación de usabilidad y sobre como aplicarlas.

A partir del trabajo desarrollado sería muy interesante extenderlo para permitir visualizar otras estructuras de datos, fue la mejora más pedida durante las pruebas con usuarios. La herramienta define una base en cuanto a la manera de capturar las operaciones realizadas por el usuario, al modelo de datos y a la visualización que podría ser expandida para otras estructuras de datos. Sería particularmente interesante extender la herramienta a árboles, ya que junto con las listas enlazadas es la estructura de datos que más se estudia en los cursos básicos de Ciencias de la Computación. Otras estructuras de datos que también sería interesante agregar son: grafos, arreglos, listas circulares, tablas de hash y listas doblemente enlazadas.

Otro trabajo futuro que se podría realizar es aprovechar la representación intermedia que utiliza la herramienta para crear más front-ends o más back-ends. Por ejemplo, utilizando el modelo actual se podrían implementar front-ends que generen visualizaciones utilizando GraphViz y también se podría implementar back-ends en otros lenguajes, cómo por ejemplo JavaScript.
