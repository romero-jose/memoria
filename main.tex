\documentclass{umemoria}
\usepackage{minted}
\usepackage{booktabs}
\usepackage{subfig}
\usepackage{svg}
\usepackage[]{biblatex}

\renewcommand{\listingscaption}{Código}
\renewcommand{\listoflistingscaption}{Índice de Códigos}

\addbibresource{bibliografia.bib}

%% Uncomment for final version:
% \usepackage{microtype}

\defaultfontfeatures{Scale=MatchLowercase}
\defaultfontfeatures[\rmfamily]{Ligatures=TeX,Scale=1}
\setmainfont{Minion 3}[
  Numbers={Proportional,Lining},
  UprightFeatures={
    SizeFeatures={
      {Size={-9},Font=*Caption},
      {Size={9-14},Font=*Regular},
      {Size={14.0-24},Font=*Subhead},
      {Size={24-},Font=*Display}
  }},
  ItalicFeatures={
    SizeFeatures={
      {Size={-9},Font=*Caption Italic},
      {Size={9-14},Font=*Italic},
      {Size={14.0-24},Font=*Subhead Italic},
      {Size={24-},Font=*Display Italic}
  }},
  BoldFeatures={
    SizeFeatures={
      {Size={-9},Font=*Caption Bold},
      {Size={9-14},Font=*Bold},
      {Size={14.0-24},Font=*Subhead Bold},
      {Size={24-},Font=*Display Bold}
  }},
  BoldItalicFeatures={
    SizeFeatures={
      {Size={-9},Font=*Caption Bold Italic},
      {Size={9-14},Font=*Bold Italic},
      {Size={14.0-24},Font=*Subhead Bold Italic},
      {Size={24-},Font=*Display Bold Italic}
  }},
  SmallCapsFeatures = {
    Letters = SmallCaps,
    LetterSpace = 10,
  }]
\setsansfont[]{Helvetica Neue}
\setmonofont[]{JetBrains Mono}

\depto{Departamento de Ciencias de la Computación}
\author{José Luis Romero Munizaga}
\title{Librería de Visualización de Estructuras de Datos}

% incluir ambos comandos para una doble titulación
%  o quitar el comando que no aplica
\memoria{Ingeniero Civil en Computación}
% \tesis{Magíster en ???}
%\tesis{Doctor en ???} % incluir solo este comando para doctorados

% puede haber varios profesores guía seperados por coma;
% pero si es una memoria, solo puede haber un profesor guía
\guia{Ivan Anselmo Sipirán Mendoza}

% puede haber varios profesores co-guía seperados por coma;
% pero si es una memoria, el profesor co-guía será el primer
% integrante de la comisión
%\coguia{Nombre Completo Co-Guía} % incluir en caso de co-guía de *tesis*

%\cotutela{Nombre Institución} % incluir en caso de cotutela

\comision{Nombre Completo Uno,Nombre Completo Dos,Nombre Completo Tres}

%\auspicio{Nombre Institución} % incluir en caso de recibir financiamiento

% tiene que ser el año en que se da el examen de título/grado (defensa)
%\anho{2021} % incluir solo para reemplazar el año actual

\usepackage{lipsum}

\begin{document}

\frontmatter
\maketitle

\begin{resumen}
\end{resumen}

% opcional: incluir para tesis en inglés;
%  en este caso hay que tener el resumen y abstract
%   en ambos idiomas
%\begin{abstract}
%\lipsum[1-4]
%\end{abstract}

\begin{dedicatoria}
Una dedicatoria corta.
\end{dedicatoria}

\begin{thanks}
\end{thanks}

\tableofcontents
\listoftables % opcional
\listoffigures % opcional

\mainmatter

\chapter{Introducción}

Posiblemente uno de los temas más relevantes en las Ciencias de la Computación corresponde a las Estructuras de Datos. Las estructuras de datos son maneras de almacenar y organizar los datos para facilitar su acceso y modificación~\cite{Algorithms}. Comúnmente se utilizan diagramas y otras visualizaciones para facilitar la comprensión de ellas y las operaciones que se realizan sobre ellas. En particular, las visualizaciones de estructuras de datos pueden ser útiles para la docencia en Ciencias de la Computación y pueden ser utilizadas tanto por los estudiantes como por los profesores. Las visualizaciones de algoritmos son efectivas para ayudar en el aprendizaje de estudiantes de Ciencias de la Computación~\cite{Hundhausen2002}.

Tradicionalmente estas visualizaciones han sido diagramas estáticos, pero como las estructuras de datos son dinámicas puede ser útil contar con visualizaciones animadas. Además, las visualizaciones animadas suelen estar separadas del ambiente donde se implementan las estructuras de datos y podría ser útil que estas animaciones estén integradas en el ambiente de desarrollo.

El ambiente de desarrollo interactivo Jupyter Notebooks es muy popular y es una buena opción para crear esta herramienta porque permite mostrar elementos interactivos y animados dentro del mismo ambiente de desarrollo. El lenguaje más utilizado en este ambiente es Python el cual además es el segundo lenguaje de programación más utilizado según una encuesta realizada por GitHub~\cite{encuesta-github}.

En particular, en el curso CC3001 Algoritmos y Estructuras de Datos de la Facultad de Ciencias Físicas y Matemáticas (FCFM) de la Universidad de Chile se enseñan estructuras de datos y se utiliza Python en conjunto con Jupyter Notebooks. Para este curso se desarrolló una herramienta para ayudar en la docencia que genera visualizaciones estáticas. Sin embargo, esta herramienta tiene varias limitaciones y sería útil contar con una herramienta que solucione estos problemas y permita generar visualizaciones animadas. 

Para lograr esto se implementó la herramienta \textit{dsvisualizer}, una librería de Python que genera visualizaciones animadas de estructuras de datos implementadas por los usuarios en Jupyter Notebooks. En su versión actual se limita a listas enlazadas, pero está diseñada para poder expandirla a otras estructuras de datos. La herramienta está implementada como un Jupyter Widget con un back-end escrito en Python que se encarga de registrar las operaciones realizadas por el usuario y un front-end escrito en JavaScript que genera las visualizaciones usando las operaciones registradas por el back-end. Las dos partes de la herramienta se comunican usando la librería ipywidgets, que usa tecnologías web para mantener modelos sincronizados en el front-end y el back-end.

Para validar la herramienta se realizaron pruebas con estudiantes de la FCFM. Durante estas los usuarios probaron la herramienta y luego contestaron un cuestionario. La primera parte del cuestionario es el formulario System Usability Scale (SUS) ---un método estándar para evaluar usabilidad de distintos tipos de herramientas--- y la segunda parte del formulario está compuesta por preguntas abiertas sobre la herramienta.

\section{Objetivos}

\subsection*{Objetivo General}\label{sec:obj-g}

Diseñar e implementar una librería para generar visualizaciones de estructuras de datos en Notebooks de Python que sea efectiva, eficiente y usable.

\subsection*{Objetivos Específicos}\label{sec:obj-e}

\begin{enumerate}
  \item Diseñar e implementar el widget para que sea fácil de usar.
  \item Diseñar e implementar el modelo del widget.
  \item Implementar la visualización de la estructura de datos a partir del modelo del widget.
  \item Evaluar la utilidad y la usabilidad de la librería.
\end{enumerate}

\chapter{Estado del Arte}

\section{Librerías de visualización de estructuras de datos}

Actualmente existen varias librerías para generar visualizaciones de estructuras de datos, pero muchas de ellas están implementadas en lenguajes distintos de Python y no están diseñadas para poder ser utilizadas en Notebooks.
Un ejemplo de esto es la librería Reftree~\cite{Stanch2021} que permite crear visualizaciones de estructuras de datos, pero en el lenguaje Scala, por lo tanto, no se puede utilizar para visualizar estructuras de datos implementadas en Python y tampoco se puede utilizar en Jupyter Notebooks.
Otro ejemplo es la librería Lolviz~\cite{Lolviz} que, si bien si permite visualizar en Notebooks estructuras de datos implementadas en Python, cuenta con la limitación de que no puede generar animaciones. En la figura \ref{fig:comparacion} se pueden ver ejemplos de las visualizaciones generadas por estas librerías.

\begin{figure}[htb]%
    \centering
    \subfloat[Reftree]{
        \centering
        \includegraphics[width=\dimexpr(\linewidth-12pt)/2\relax, height=\dimexpr(\linewidth-12pt)/2\relax]{imagenes/ejemplos/reftree}
        \label{fig:reftree}
    }
    \subfloat[Lolviz]{
        \centering
        \includegraphics[width=\dimexpr(\linewidth-12pt)/2\relax]{imagenes/ejemplos/lolviz}
        % \vspace{38px}
        \label{fig:lolviz}
    }\\
    \subfloat[Aed Utilities]{
        \centering
        \includesvg[width=\dimexpr(\linewidth-12pt)/2\relax]{imagenes/ejemplos/aed}
        \label{fig:aed}
    }
    \subfloat[Manim]{
        \centering
        \includegraphics[width=\dimexpr(\linewidth-12pt)/2\relax]{imagenes/ejemplos/manim}
        \label{fig:manim}
    }
    \caption[Ejemplos de las distintas herramientas.]{Ejemplos de visualizaciones generadas por las distintas herramientas. Obtenidas de \cite{Stanch2021}, \cite{Lolviz}, \cite{aed-utilities} y \cite{manim}, respectivamente.}
    \label{fig:comparacion}
\end{figure}

Por otro lado, existe la librería Aed-Utilities~\cite{aed-utilities} creada para el curso de Algoritmos y Estructuras de Datos de la FCFM por el Profesor Ivan Sipiran; que, si bien permite generar diagramas de estructuras de datos en Python y mostrarlas en un Notebook, cuenta con las siguientes limitaciones: no permite crear animaciones, la API no es muy cómoda de utilizar y el algoritmo que utiliza para generar las visualizaciones es poco eficiente. % TODO: Explicar

Tanto Aed-Utilites como Lolviz generan los diagramas utilizando Graphviz ---una herramienta originalmente desarrollada por AT\&T para dibujar gráficos especificados en el lenguaje DOT--- que permite generar diagramas de muy buena calidad, pero no permite crear animaciones.

En cuanto a las herramientas para generar animaciones, existe la librería Manim~\cite{manim}, diseñada para generar visualizaciones animadas de matemáticas. Si bien no está diseñada para crear visualizaciones de estructuras de datos, es relevante porque permite generar animaciones y estas animaciones se pueden ver desde Notebooks. Esta librería tiene un diseño orientado a objetos, donde una animación es un objeto que tiene un campo con el objeto matemático que representa y tiene un método que permite animar el objeto según una función de interpolación. Para generar el resultado final puede utilizar varios back-ends de rendering, incluyendo OpenGL y WebGL. Utilizando este último cuando se usa desde un Notebook. En la figura \ref{fig:manim} se puede ver una visualización de una función creada con esta herramienta.

Otra herramienta enfocada en las matemáticas es Penrose~\cite{Penrose}, una herramienta que permite crear diagramas a partir de un programa en un lenguaje específico de dominio. La disposición de los elementos en el diagrama se obtiene mediante optimización numérica. El lenguaje específico de dominio (o DSL por sus iniciales en inglés) se separa en un archivo que define solo la parte matemática y en otro que define el estilo.

Dado que no se encontró ninguna herramienta existente que permita visualizar estructuras de datos implementadas en Python, permita generar animaciones y permita mostrar las visualizaciones generadas en un Jupyter Notebooks, se requiere crear una nueva librería que implemente esta funcionalidad.

\section{Jupyter Notebooks}

Los Jupyter Notebooks son archivos interactivos que almacenan bloques de texto, código y también pueden contener imágenes, videos y animaciones interactivas. Un Jupyter Notebook tiene varios componentes que permiten construir la experiencia de usuario. En primer lugar, está el archivo del notebook que es un archivo JSON con la extensión \texttt{.ipynb}, que almacena en el disco todos los datos persistentes del notebook, incluyendo los bloques de texto, los bloques de código y sus salidas, las imágenes, etc. Después, está el \textit{kernel} o núcleo, que es un intérprete del lenguaje respectivo, con la capacidad de comunicarse con el servidor del notebook. El servidor del notebook se encarga de la comunicación entre el front-end, el kernel y el archivo del notebook. Finalmente, está el front-end del notebook, que se encarga de mostrarle al usuario una representación del notebook y le permite a este interactuar con el notebook, usualmente corresponde a un navegador web. En la figura \ref{fig:notebook_arq} se puede ver un diagrama de esta arquitectura.

\begin{figure}[htb]
  \centering
  \includegraphics[width=0.8\linewidth]{imagenes/notebook/notebook_components}
  \caption[Arquitectura de Jupyter Notebooks]{Arquitectura de Jupyter Notebooks. Obtenido de \cite{arq-notebook}.}
  \label{fig:notebook_arq}
\end{figure}

Existen múltiples kernels, uno por cada lenguaje que soportan los Jupyter Notebooks, pero el más conocido es el kernel IPython que es el kernel del lenguaje Python. IPython además de ser un kernel de Jupyter Notebooks es un intérprete interactivo de Python, con funcionalidades como contenido multimedia y completado de comandos. Cuando se utiliza un notebook con el lenguaje Python, el servidor del notebook se comunica con este intérprete que es que se encarga de interpretar los bloques de código y responder con la salida generada cuando el servidor del notebook envía el request.

Para crear componentes interactivos en un Notebook de Python se utiliza la librería Jupyter Widgets. Esta librería permite definir un modelo que representa el componente, y una vista en JavaScript y HTML que es mostrada por el front-end del Notebook. El modelo se define tanto en JavaScript como en Python y la librería se encarga de mantener las dos versiones sincronizadas. Para mantener las dos versiones sincronizadas, es necesario serializar y deserializar los campos del modelo, para poder enviarlos en formato JSON. Entonces, para los campos sencillos, la librería se puede encargar de la serialización o deserialización, pero para los casos más complejos se deben definir manualmente los serializadores y deserializadores para cada campo. En la figura \ref{fig:widget_arq} se observa la arquitectura de un widget generado con esta librería.

\begin{figure}[htb]
  \centering
  \includegraphics[width=0.8\textwidth]{imagenes/notebook/WidgetModelView}
  \caption[Arquitectura de Jupyter Widgets]{Arquitectura de Jupyter Widgets. Obtenido de \cite{arq-widget}.}
  \label{fig:widget_arq}
\end{figure}

\section{D3}

D3 es una librería de JavaScript para manipular documentas en función de datos. Esta librería permite manipular el DOM (Document Object Model) y aplicarle transformaciones según los datos. Permite usar las tecnologías estándar de la web para generar visualizaciones de buena calidad.

Es particularmente útil la capacidad de D3 de manipular elementos SVG (Scalable Vector Graphics) ---un lenguaje de markup para describir gráficos en dos dimensiones---. Esto permite generar animaciones en dos dimensiones de forma relativamente simple. Además esta técnica es bastante eficiente gracias a que los navegadores implementan motores muy eficientes para la renderización de elementos SVG.

Para lograr esto D3 permite enlazar datos con elementos del DOM y definir los atributos de estos elementos como función de los datos. Además, cuando cambian los datos se puede hacer una transición suave de los elementos correspondientes a su nueva posición según la función definida anteriormente. La figura~\ref{fig:d3-ejemplo} muestra un ejemplo de una visualización generada con D3 que representa un conjunto de datos como una agrupación de círculos.

\begin{figure}[htb]
    \centering
    \includegraphics[width=0.8\textwidth]{imagenes/d3/ejemplo.png}
    \caption{Ejemplo de visualización generada con D3}
    \label{fig:d3-ejemplo}
\end{figure}


\chapter{Problema}

%%%%%%%%%%%%%%%%%%%%%%%%%%%%%%%%%%%%%%%%%%%%%%%%%%%%%%%%%%%%%%%%%%%%%%%%%%%%%%%%%
%%% TODO: Expandir esta sección, por ahora es lo mismo que en la introducción %%%
%%%%%%%%%%%%%%%%%%%%%%%%%%%%%%%%%%%%%%%%%%%%%%%%%%%%%%%%%%%%%%%%%%%%%%%%%%%%%%%%%

El problema abordado consiste en implementar una herramienta que permita generar visualizaciones animadas de las operaciones que se realizan sobre una estructura de datos implementada por el usuario. Para esto la herramienta debe ser capaz de registrar las operaciones sobre la estructura y debe ser capaz de generar una visualización animada a partir de las operaciones que registró. Es deseable que la visualización se genere en el mismo notebook y que el usuario tenga que agregar la menor cantidad de instrumentación a su código para que la herramienta funcione.

Implementar una librería que cumpla con estos requerimientos sería beneficioso para estudiantes de ciencias de la computación, profesores o personas en general, que estén aprendiendo estructuras de datos, o que deseen generar visualizaciones de estructuras de datos implementadas en Python y verlas en Jupyter Notebooks. Además, concretamente permitiría contribuir a la docencia en el curso Algoritmos y Estructuras de Datos de la FCFM.

%%%%%%%%%%%%%%%%%%%%%%%%%%%%%%%%%%%%%%%%%%%%%%%%%%%%%%%%%%%%%%%%%%%%%%%%%%%%%%%%%

\chapter{Solución}

\section{Arquitectura}

La solución es una librería de Python para generar visualizaciones animadas de estructuras de datos y las operaciones que se realizan sobre estas, que pueda ser usada en Jupyter Notebooks.

La librería le permite al usuario implementar una estructura de datos y, agregando la instrumentación provista por la librería, le permite generar una visualización animada de las operaciones que se realizaron sobre la estructura.

Para permitir esto la librería está compuesta por dos partes: el \textit{back-end} y el \textit{front-end}. Estas dos partes se comunican entre sí utilizando un modelo de datos común que cada una de las partes sabe serializar y deserializar.

\begin{figure}[htb]
    \centering
    \includesvg[width=\textwidth]{imagenes/diagramas/arquitectura.svg}
    \caption{Diagrama de la arquitectura}
    \label{fig:diagrama-arq}
\end{figure}

El \textit{back-end}, implementado en Python, define la instrumentación para capturar las operaciones realizadas sobre la estructura de datos. Se encarga de mantener un registro de todas las operaciones que se realizan sobre esta. Este registro se mantiene utilizando una representación de las operaciones primitivas sobre la estructura de datos. Teniendo esto, cuando un usuario quiere generar la visualización, este registro de operaciones es serializado y enviado al \textit{front-end}. Esto se logra utilizando usando la librería IPython Widgets en combinación con el serializador definido para el modelo.

El \textit{front-end}, implementado en TypeScript, recibe el modelo serializado, deserializa el modelo y genera la visualización a partir de este. Para generar la visualización utiliza la librería D3js, que permite manipular el DOM (Document Object Model) en función de los datos del modelo.

La figura~\ref{fig:flujo-informacion} es un diagrama que muestra el flujo de la información cuando se utiliza la herramienta. Primero el usuario implementa y usa la estructura de datos en el notebook, después a partir de esto se genera un registro en el back-end, luego este registro se sincroniza con el front-end y finalmente este último genera la animación en el navegador.

\begin{figure}[htb]
    \centering
    \includesvg[pretex=\footnotesize,width=\textwidth]{imagenes/diagramas/diagrama-de-flujo.svg}
    \caption{Flujo de la información}
    \label{fig:flujo-informacion}
\end{figure}

En la figura~\ref{fig:diagrama-arq} se puede ver un diagrama que representa la arquitectura física de la solución. El usuario interactúa con la herramienta usando el navegador, donde corre el front-end. Este se comunica usando HTTP y Websockets con el servidor de Notebooks, que se encarga de interactuar con el archivo del Notebook y se comunica con el kernel usando ZeroMQ ---una librería de comunicación asíncrona orientada a mensajes---. El kernel contiene el intérprete de Python y es donde corre el back-end de la herramienta.

\section{Modelo de datos}

Cuando un usuario quiere generar una visualización el back-end le envía al front-end una representación de las operaciones realizadas sobre la estructura de datos. Para esto se diseñó un modelo que representa las operaciones primitivas que se pueden realizar sobre las listas enlazadas. Además, el modelo contiene metadatos que son útiles para generar la visualización.

El modelo que se utiliza para generar la visualización consiste en una lista de operaciones y metadatos de la visualización. Cada operación consiste en una operación primitiva y metadatos de la operación.

Las operaciones primitivas son las siguientes:
\begin{itemize}
    \item{\makebox[3.5cm]{Init(id, value, next)\hfill}}: Inicializar un nodo
    \item{\makebox[3.5cm]{SetValue(id, value)\hfill}}: Asignar el valor de un nodo
    \item{\makebox[3.5cm]{GetValue(id)\hfill}}: Obtener el valor de un nodo
    \item{\makebox[3.5cm]{SetNext(id, next)\hfill}}: Asignar el siguiente de un nodo
    \item{\makebox[3.5cm]{GetNext(id)\hfill}}: Obtener el siguiente de un nodo
\end{itemize}
donde \textit{id} es el identificador único de cada nodo, \textit{value} es una cadena de texto que representa el valor del nodo y \textit{next} es el identificador del siguiente nodo en la lista enlazada o es \textit{null}.

Los metadatos de la visualización son configuraciones de la visualización, por ejemplo, la duración de las transiciones y la duración de los fade-ins y fade-outs. En cambio, los metadatos de las operaciones contienen información que solo es relevante para esa operación, por ejemplo, si se debe animar o no esa operación y el código fuente que dio origen a la operación.

En la figura~\ref{fig:codigo-vs-modelo} se puede ver un ejemplo de código fuente que usa la herramienta, una captura de la animación generada a partir de ese código y el modelo generado a partir del código que se usa para crear la visualización.

\begin{figure}[p]
    \centering
    \begin{subfigure}[b]{0.49\textwidth}
        \centering
        \begin{minted}[linenos=false,fontsize=\scriptsize]{python}
@node('value', 'next')
class Node:
    def __init__(self, value, next=None):
        self.value = value
        self.next = next

@container(lines_before=0, lines_after=0)
class List:
    def __init__(self):
        self.node = None
        
    def append(self, value):
        node = self.node
        if node is None:
            self.node = Node(value, None)
            return

        next = node.next
        while next is not None:
            node = next
            next = next.next
        node.next = Node(value, None)

l = List()
l.append(1)
l.append(2)
l.visualize()
        \end{minted}
        \caption{Código}
        \label{fig:codigo-vs-modelo:codigo}
    \end{subfigure}
    \hfill
    \begin{subfigure}[b]{0.49\textwidth}
        \centering
        \includegraphics[width=\textwidth]{imagenes/codigo-imagen-modelo.png}
        \caption{Captura de la visualización}
        \label{fig:codigo-vs-modelo:imagen}
    \end{subfigure}
    \begin{subfigure}[b]{0.8\textwidth}
        \centering
        \begin{minted}[linenos=false,fontsize=\scriptsize]{json}
{
  "operations": [
    {
      "operation": { "operation": "init", "id": 6, "value": "1", "next": null },
      "metadata": {
        "animate": true,
        "source": ["l.append(1)\n", "> 15     self.node = Node(value, None)\n"]
      }
    },
    {
      "operation": { "operation": "get_next", "id": 6 },
      "metadata": {
        "animate": true,
        "source": ["l.append(2)\n", "> 18 next = node.next\n"]
      }
    },
    {
      "operation": { "operation": "init", "id": 7, "value": "2", "next": null },
      "metadata": {
        "animate": true,
        "source": ["l.append(2)\n", "> 22 node.next = Node(value, None)\n"]
      }
    },
    {
      "operation": { "operation": "set_next", "id": 6, "next": 7 },
      "metadata": {
        "animate": true,
        "source": ["l.append(2)\n", "> 22 node.next = Node(value, None)\n"]
      }
    }
  ],
  "metadata": { "transition_duration": 1000, "fade_in_duration": 1000 }
}
                      
        \end{minted}
        \caption{Modelo generado a partir del código}
        \label{fig:codigo-vs-modelo:modelo}
    \end{subfigure}
    \caption{Código, captura de la visualización y modelo}
    \label{fig:codigo-vs-modelo}
\end{figure}

Este modelo en realidad representa un grafo dirigido con un grado máximo de 1. Este nivel de flexibilidad es necesario porque como la representación se genera a partir de un programa escrito por el usuario, este puede contener errores que hagan que la estructura de datos implementada no sea necesariamente una lista enlazada. Por ejemplo, por ejemplo, el usuario puede crear nodos que no estén conectados entre sí y puede crear ciclos.

Para una versión futura de la herramienta se podría cambiar esta representación para que represente grafos sin un grado máximo. Esto permitiría representar estructuras de datos tales como grafos, árboles y listas enlazadas. Se consideró utilizar esta representación desde un principio, pero no se hizo porque dificultaba la generación de la visualización para la estructura de datos abordada.

\section{Back-end}

El back-end se encarga de proveer la instrumentación necesaria para que el usuario pueda generar visualizaciones de las estructuras de datos que ha implementado. Usando esta instrumentación mantiene un registro de las operaciones primitivas realizadas sobre la estructura de datos y cuando el usuario quiere visualizar el resultado serializa este registro y lo envía al \textit{front-end}.

Para esto la librería provee dos \textit{decoradores} ---azúcar sintáctica de Python para aplicar funciones a las definiciones de clases o funciones--- \textit{node} y \textit{container}. El primero de estos denota la clase que representa el nodo de la lista enlazada, mientras que el segundo denota la clase que contiene una referencia al primer nodo de la lista. En el código~\ref{lst:ejemplo-uso} se puede ver un ejemplo del uso de estos decoradores y en la figura~\ref{fig:visualizacion_ej} se puede ver un cuadro de la visualización generada.

\begin{listing}[htb]
\caption{Ejemplo de uso de la librería}
\label{lst:ejemplo-uso}
\begin{minted}[linenos=true]{python}
from dsvisualizer import node, container

@node('hd', 'tl')
class Node():
    def __init__(self, hd, tl):
        self.hd = hd
        self.tl = tl

@container()
class List():
    def __init__(self):
        self.head = None

    def push(self, v):
        self.head = Node(v, self.head)

l = List()
l.push(1)
l.push(2)
l.push(3)
l.visualize()
\end{minted}
\end{listing}

\begin{figure}[htb]
    \centering
    \includegraphics[width=\linewidth]{imagenes/ejemplos/ejemplo}
    \caption{Captura de la visualización generada a partir del código~\ref{lst:ejemplo-uso}.}
    \label{fig:visualizacion_ej}
    \centering
\end{figure}

Para registrar las operaciones primitivas sobre la estructura de datos el decorador \textit{node} modifica los campos pasados como parámetros para que al momento de acceder o asignar a estos campos se registre la operación en el \textit{logger}.

El \textit{logger} es un objeto que mantiene el registro de las operaciones primitivas sobre la estructura de datos. Además, se puede utilizar como un \textit{context manager} (objetos de Python que definen un contexto y se utilizan con \texttt{with}) y dentro del scope introducido todas las operaciones serán registradas en este \textit{logger}. En el código~\ref{lst:ejemplo-logger-ctx-mgr} se puede ver un ejemplo de esta funcionalidad.

\begin{listing}[htb]
\caption{Ejemplo de uso del \textit{logger} como un \textit{context manager}.}
\label{lst:ejemplo-logger-ctx-mgr}
\begin{minted}[linenos=true]{python}
from dsvisualizer import logger

with Logger() as logger:
    n = Node(5, Node(10, Node(20, None)))
logger.visualize()

with logger:
    n = Node(10, n)
logger.visualize()
\end{minted}
\end{listing}

Cuando se aplica el decorador \textit{container} a una clase, se crea un Logger asociado a cada instancia de esa clase y para todos los métodos de esa clase se utiliza ese logger como \textit{context manager} para que las operaciones sobre la estructura de datos queden registradas en el logger del contenedor.

Esta parte originalmente se implementó usando herencia en vez de decoradores, pero se cambió debido a que de esta manera eran menores los cambios que el usuario debía hacer a su programa para permitirle usar la herramienta.

\section{Front-end}

Como la librería registra las operaciones primitivas no es trivial generar una visualización a partir del modelo, ya que las operaciones registradas no necesariamente representan una operación estándar sobre listas. De hecho estas operaciones definen un grafo dirigido con un grado máximo de 1. Esta representación resulta útil porque permite representar implementaciones incorrectas de una lista enlazada.

Para visualizar por separado las componentes conexas del grafo se deben encontrar los subgrafos conexos a partir de esta representación. En la literatura existen varios algoritmos para resolver este problema, pero por simplicidad se decidió acotar la librería al caso donde el grafo es acíclico, es decir, un bosque. De esta manera, todos los nodos que no tienen ningún arco que apunte a ellos son puntos de entrada. Además, todos estos puntos de entrada definen una lista conexa que no tiene conexiones con otras listas. Entonces, podemos visualizar cada una de estas componentes conexas como una lista.

Para generar la animación se visualiza en orden cada operación primitiva que se obtiene del \textit{front-end} (animándola o no dependiendo de los metadatos de la operación). Para cada operación primitiva primero se obtienen las listas conexas usando el algoritmo descrito previamente, luego se genera una lista donde cada elemento tiene la información asociada a un nodo y finalmente usando D3js se anima la transición desde la versión previa de esta lista hasta la actual.

Los datos asociados a cada nodo son: el índice de la lista a la que pertenece, el largo de la lista a la que pertenece, su índice dentro de la lista a la que pertenece, su identificador único y su valor.

Teniendo la lista, usando D3js se asocian los nodos de la visualización a cada elemento en esta lista usando el identificador del elemento. La animación tiene dos fases: primero se anima la actualización de los nodos existentes y luego se anima la entrada de nuevos nodos.

En la primera fase se transiciona con una animación desde la posición anterior de los nodos a la posición dada por los datos actualizados asociados a cada nodo y se actualizan los valores que hayan cambiado. En la segunda fase se hace un \textit{fade in} de los nuevos nodos en las posiciones correspondientes. Es importante el orden de estas fases porque si se hace en otro orden los nodos entrantes podrían aparecer detrás de nodos preexistentes.

\section{Integración Continua y Entrega Continua}

Para que la herramienta pueda ser utilizada instalándola con el administrador de paquetes de Python se publicó en el repositorio oficial del administrador de paquetes de Python, PyPi (Python Package Index)\footnote{El paquete de PyPi se encuentra disponible en \url{https://pypi.org/project/dsvisualizer/}}. Adicionalmente, cómo el front-end de la herramienta está implementado en TypeScript también fue necesario publicar un paquete en el repositorio NPM (Node Package Manager)\footnote{El paquete de NPM se encuentra disponible en \url{https://www.npmjs.com/package/jupyter-dsvisualizer}}.

En el caso de Python generar el paquete es relativamente sencillo, porque solo utiliza el código definido en el archivo \texttt{setup.py}. En cambio, en el caso del TypeScript esto involucra el uso de una serie de herramientas. Algunas de estas herramientas son: el compilador de TypeScript (TSC) para el chequeo de tipos y la generación del código en JavaScript; y Webpack para generar \textit{bundles} minimizadas a partir del código en JavaScript obtenido de TSC. La generación de \textit{bundles} minimizadas es importante porque reduce el tamaño del paquete que tiene que instalar el usuario, lo que es especialmente relevante cuando este no cuenta con una buena conexión a internet, y porque reduce el tiempo inicial que el navegador se demora en poder ejecutar el código.

Para facilitar el lanzamiento de nuevas versiones de la herramienta el repositorio se encuentra en GitHub y utiliza Github Actions ---una funcionalidad de GitHub que permite ejecutar programas después de eventos como commits o deploys--- para correr las pruebas y para publicar versiones nuevas de la librería.

Cada vez que se agrega un commit al repositorio, una GitHub Action corre los tests y los \textit{linters} de Python (Black) y de JavaScript (Prettier). Además, cuando se hace un release del repositorio otra Github Action compila los paquetes de Python y de JavaScript y los sube a los repositorios de los administradores de paquetes respectivos. Esto permite lanzar nuevas versiones de la herramienta de forma muy expedita, facilitando y acelerando el desarrollo.

\chapter{Validación}

Para evaluar la solución se hicieron pruebas con usuarios y se les pidió que contestaran un cuestionario. Los participantes fueron estudiantes de la Facultad de Ciencias Físicas y Matemáticas que accedieron a ser parte de la investigación. Estos fueron reclutados usando el foro institucional, el foro del curso CC3001 y el grupo de Telegram de los estudiantes del Departamento de Ciencias de la Computación.

Para realizar estas pruebas se le pidió la autorización al comité de ética y todos los usuarios firmaron un consentimiento informado antes de participar. En el anexo~\ref{anexo:certificacion-comite-de-etica-y-bioseguridad} se puede ver la certificación del Comité de Ética y Bioseguridad para la Investigación de la Facultad de Ciencias Físicas y Matemáticas de la Universidad de Chile. Adicionalmente, en el anexo~\ref{anexo:consentimiento} se puede ver el consentimiento informado que fue aprobado por el comité y fue firmado por los participantes.

Las pruebas con usuarios consistieron en que, a cada participante, primero se le explicó el proceso, luego se le dio una breve introducción sobre cómo usar la herramienta, después se le pidió que utilice la herramienta y, finalmente se le pidió que conteste un cuestionario.

El cuestionario tiene tres partes. La primera parte consiste en el System Usability Scale (SUS) ---un método estándar para evaluar la usabilidad de sistemas informáticos---, la segunda parte corresponde a preguntas abiertas sobre la herramienta, y la tercera parte corresponde a una caracterización del participante. En el anexo~\ref{anexo:cuestionario} se puede ver el cuestionario que fue utilizado.

La escala SUS fue introducida en~\cite{brooke1996quick} como un método sencillo para evaluar la usabilidad de sistemas informáticos y se ha vuelto un método estándar en la industria para medir la usabilidad de todo tipo de sistemas, en la sección~\ref{state-of-the-art:sus} se describe con más detalle esta escala. En~\cite{evaluation-of-sus} analizan 10 años de datos de SUS y encuentran que esta escala es una herramienta altamente robusta y versátil para las evaluaciones de usabilidad. Como la versión original está en inglés, se utilizó la versión en español adaptada y validada en~\cite{spanish-sus}, con una modificación. Esta consistió en que se cambió ``\textit{Me sentí muy confiado al usar la herramienta}'' por ``\textit{Me sentí muy confiado o confiada al usar la herramienta}'' con el objetivo de hacer el cuestionario más inclusivo.

Las preguntas abiertas que se le hicieron los participantes fueron: ``\textit{¿Qué te gustó de la herramienta? ¿Por qué?}'', ``\textit{¿Qué no te gustó de la herramienta? ¿Por qué?}'', y ``\textit{¿Cómo podría ser mejor la herramienta?}'' Se escogieron estas preguntas para tener una idea de porque a los usuarios les gustó o no la herramienta y también de cómo esta se podría mejorar en el futuro.

En la caracterización se preguntó por edad, género y avance curricular. La edad se preguntó en intervalos de 2 años entre los 18 años y 30 o más años. Se preguntó de esta manera para evitar poder identificar a los participantes a partir de sus respuestas. Para el género las opciones eran \textit{hombre}, \textit{mujer}, \textit{otro} y \textit{prefiero no responder}. Para el avance curricular se les pidió a los estudiantes que marcaran todos los cursos que han cursado o estaban cursando de una lista de 5 cursos obligatorios de Ingeniería Civil en Computación. Cómo los cursos seleccionados requieren tomar el curso anterior en la lista, esta respuesta se puede representar con un número del 0 al 5, donde el número representa que el estudiante ha tomado todos los cursos de la lista hasta ese número.

Se preguntó por edad y género porque en~\cite{evaluation-of-sus} se observa cierta relación entre estos factores y el puntaje obtenido en SUS. Se preguntó por avance curricular porque se cree que puede haber una relación entre el nivel de conocimiento en programación o en ciencias de la computación y la facilidad para usar la herramienta.

En total participaron 12 estudiantes. De estos, 10 se identificaron como hombres, 1 como mujer y 1 como otro. La edad promedio de los participantes fue 22 años\footnote{El promedio de las edades se calculó utilizando la marca de clase de los intervalos correspondientes}. En cuanto al avance curricular, 9 de los estudiantes marcaron 2, lo que significa que han cursado los cursos de la lista hasta Algoritmos y Estructuras de Datos. En la figura~\ref{fig:genero} se puede ver la distribución de género y en la tabla~\ref{tab:resumen-resultados} se puede ver el promedio y desviación estándar de la edad y avance curricular.

\begin{figure}[hbt]
    \centering
    \begin{tikzpicture}
        \pie{83.3/Hombre,
              8.3/Mujer,
              8.3/Otro}
    \end{tikzpicture}
    \caption{Distribución de género}
    \label{fig:genero}
\end{figure}

Las respuestas de los participantes se pueden ver en la tabla~\ref{tab:respuestas}. Se muestra el avance, edad, el puntaje por pregunta para cada pregunta y el puntaje total, la última fila muestra el promedio para cada columna.
Cómo el cuestionario SUS alterna afirmaciones positivas con negativas, para facilitar la compresión de los resultados se presentan las respuestas, de forma que los puntajes están en una escala entre 0 y 4, donde 0 es lo peor y 4 es lo mejor. Utilizando el sistema de puntuación de SUS se obtuvo un puntaje promedio de 90.0, con una desviación estándar de 7.6.

\begin{table}[hbt]
    \centering
    \caption{Resumen de los resultados}
    \label{tab:resumen-resultados}
    \begin{tabular}{@{}lrr@{}}
    \toprule
                      & \textbf{Promedio} & \textbf{Desviación} \\ \midrule
    Puntaje           & 90.0              & 7.6        \\
    Avance curricular & 3.3               & 1.2        \\
    Edad              & 22                & 1.5        \\ \bottomrule
    \end{tabular}
\end{table}

Para interpretar este puntaje es preferible calcular el percentil en el que quedaría este puntaje dentro de un grupo grande de evaluaciones realizadas utilizando la escala SUS. Por esto, se utilizó la tabla de~\cite{quantifying-the-user-experience} para transformar el puntaje a un percentil, obteniendo que el puntaje se encuentra en el percentil 99.8.

\begin{table}[hbt]
    \caption[Matriz de respuestas]{Matriz de respuestas, transformando los puntajes de los ítems a una escala entre 0 y 4 donde mayor es mejor.}
    \label{tab:respuestas}
    \centering
    \begin{tabular}{@{}ccccccccccccr@{}}
    \toprule
    \textbf{Avance} & \textbf{Edad} & \textbf{P1}  & \textbf{P2}  & \textbf{P3}  & \textbf{P4}  & \textbf{P5}  & \textbf{P6}  & \textbf{P7}  & \textbf{P8}  & \textbf{P9}  & \textbf{P10} & \textbf{Puntaje} \\ \midrule
    2      & 20--21   & 4   & 4   & 3   & 4   & 4   & 4   & 3   & 4   & 3   & 2   & 87.5    \\
    2      & 22--23   & 4   & 4   & 4   & 3   & 4   & 4   & 4   & 4   & 4   & 3   & 95.0    \\
    5      & 24--25   & 1   & 4   & 4   & 4   & 2   & 4   & 4   & 4   & 3   & 1   & 77.5    \\
    4      & 22--23   & 1   & 4   & 4   & 1   & 4   & 4   & 4   & 4   & 2   & 3   & 77.5    \\
    2      & 20--21   & 3   & 4   & 4   & 4   & 4   & 4   & 4   & 4   & 3   & 1   & 87.5    \\
    2      & 22--23   & 4   & 4   & 3   & 3   & 3   & 4   & 3   & 3   & 3   & 4   & 85.0    \\
    2      & 22--23   & 1   & 4   & 4   & 4   & 4   & 4   & 4   & 4   & 4   & 4   & 92.5    \\
    2      & 22--23   & 4   & 4   & 4   & 4   & 4   & 4   & 4   & 4   & 4   & 4   & 100.0   \\
    2      & 20--21   & 4   & 4   & 4   & 3   & 3   & 4   & 4   & 3   & 3   & 4   & 90.0    \\
    2      & 20--21   & 4   & 4   & 4   & 4   & 4   & 4   & 4   & 4   & 4   & 4   & 100.0   \\
    2      & 20--21   & 3   & 4   & 4   & 4   & 4   & 4   & 4   & 4   & 4   & 4   & 97.5    \\
    5      & 24--25   & 4   & 3   & 4   & 4   & 4   & 4   & 3   & 4   & 3   & 3   & 90.0    \\ \addlinespace[3pt]
    2.7    & 22.0 & 3.1 & 3.9 & 3.8 & 3.5 & 3.7 & 4.0 & 3.8 & 3.8 & 3.3 & 3.1 & 90.0    \\ \bottomrule
    \end{tabular}
\end{table}

En cuanto a las preguntas abiertas. En la primera pregunta, \textit{¿Qué te gustó de la herramienta? ¿Por qué?}, las respuestas más frecuentes mencionaban que la herramienta muestra el código fuente que dio origen a la operación y que solo requiere agregar dos líneas de código para usarla. En la segunda pregunta, \textit{¿Qué no te gustó de la herramienta? ¿Por qué?}, no hubo respuestas significativamente más comunes que otras. Sin embargo, una respuesta que se repitió dos veces tiene que ver con que al eliminar un nodo de la lista este visualmente se mueve hacia debajo la lista original en vez de desaparecer. En la tercera pregunta, \textit{¿Cómo podría ser mejor la herramienta?}, la respuesta más frecuenta estaba relacionada permitir visualizar otras estructuras de datos que no sean listas enlazadas. En el anexo~\ref{anexo:comentarios} se pueden ver todas las respuestas a las preguntas abiertas.

Estos resultados muestran que la herramienta soluciona el problema abordado y cumple con el requisito de permitirle al usuario utilizar la herramienta agregando la menor cantidad de instrumentación necesaria a su código. Sin embargo, también presenta oportunidades de mejora. La principal oportunidad de mejora tiene que ver con permitir generar visualizaciones de otras estructuras de datos. Otra oportunidad de mejora sería mostrar en la visualización las referencias a los nodos del \textit{contenedor}. Esto ayudaría a los usuarios a entender qué está pasando en algunos de los casos que ocurrieron durante las pruebas con usuarios. Algunas oportunidades de mejora de menor importancia tienen que ver con hacer la herramienta más configurable, permitiendo cambiar variables como: color de fondo, color de los nodos, forma de los nodos, tamaño de los nodos, etc.

\chapter{Conclusión}

% Breve resumen del trabajo realizado

Se desarrolló \textit{dsvisualizer}, una herramienta para generar visualizaciones animadas de estructuras de datos para ser utilizada en Jupyter Notebooks. Esta herramienta permite a los usuarios visualizar sus propias implementaciones, agregando la menor cantidad de instrumentación necesaria a su código. Además, se evaluó esta herramienta realizando pruebas con 12 estudiantes de la Facultad de Ciencias Físicas y Matemáticas. El resultado de esta evaluación fue muy positivo, obteniendo un puntaje promedio de 90.0, el cual se encuentra en el percentil 99.8.

Se cumplió el objetivo general de ``\textit{Diseñar e implementar una librería para generar visualizaciones de estructuras de datos en Notebooks de Python que sea efectiva, eficiente y usable}'' y también se cumplieron todos los objetivos específicos.

Esta herramienta podrá ser utilizada cómo una ayuda para la docencia en el curso Algoritmos y Estructuras de Datos de la Facultad de Ciencias Física y Matemáticas, que beneficiará a los futuros estudiantes de esta facultad. Además, se encuentra disponible públicamente, por lo que cualquier persona que le interese podrá utilizarla para ayudarla en su entendimiento de las estructuras de datos. La herramienta cuenta con licencia BSD (Berkeley Software Distribution), lo que permite redistribuirla, usarla y modificarla con mínimas restricciones.

Durante el desarrollo de la memoria se aprendió mucho sobre distintos temas. Se aprendió sobre la generación de visualizaciones, primero utilizando Threejs y luego utilizando D3js al descubrir que Threejs no era la herramienta apropiada para este problema. Se aprendió sobre distintos mecanismos del lenguaje Python, la primera versión de la librería utilizaba herencia, luego se pasó por una versión que utilizaba metaclases y finalmente, se optó por utilizar decoradores, ya que esto permite una interfaz más cómoda para el usuario. Se aprendió sobre metodologías para la evaluación de usabilidad y sobre como aplicarlas.

A partir del trabajo desarrollado sería muy interesante extenderlo para permitir visualizar otras estructuras de datos, fue la mejora más pedida durante las pruebas con usuarios. La herramienta define una base en cuanto a la manera de capturar las operaciones realizadas por el usuario, al modelo de datos y a la visualización que podría ser expandida para otras estructuras de datos. Sería particularmente interesante extender la herramienta a árboles, ya que junto con las listas enlazadas es la estructura de datos que más se estudia en los cursos básicos de Ciencias de la Computación. Otras estructuras de datos que también sería interesante agregar son: grafos, arreglos, listas circulares, tablas de hash y listas doblemente enlazadas.

Otro trabajo futuro que se podría realizar es aprovechar la representación intermedia que utiliza la herramienta para crear más front-ends o más back-ends. Por ejemplo, utilizando el modelo actual se podrían implementar front-ends que generen visualizaciones utilizando GraphViz y también se podría implementar back-ends en otros lenguajes, cómo por ejemplo JavaScript.


% ver https://www.overleaf.com/learn/latex/Glossaries
% \input{glosario.tex} % opcional

\printbibliography[
    heading=bibintoc,
]

% opcional ...
\begin{appendices}
% \chapter{Código Fuente}
\lipsum[50-60]

\end{appendices}
\end{document}
