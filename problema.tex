\chapter{Problema}

%%%%%%%%%%%%%%%%%%%%%%%%%%%%%%%%%%%%%%%%%%%%%%%%%%%%%%%%%%%%%%%%%%%%%%%%%%%%%%%%%
%%% TODO: Expandir esta sección, por ahora es lo mismo que en la introducción %%%
%%%%%%%%%%%%%%%%%%%%%%%%%%%%%%%%%%%%%%%%%%%%%%%%%%%%%%%%%%%%%%%%%%%%%%%%%%%%%%%%%

El problema abordado consiste en implementar una herramiento que permita generar visualizaciones animadas de las operaciones que se realizan sobre una estructura de datos implementada por el usuario. Para esto la herramiento debe ser capaz de registrar las operaciones sobre la estructura y debe ser capaz de generar una visualización animada a partir de las operaciones que registró. Es deseable que la visualización se genere en el mismo notebook y que el usuario tenga que agregar la menor cantidad de instrumentación a su código para que la herramienta funcione.

Implementar una librería que cumpla con estos requerimientos sería beneficioso para estudiantes de ciencias de la computación, profesores o personas en general, que estén aprendiendo estructuras de datos, o que deseen generar visualizaciones de estructuras de datos implementadas en Python y verlas en Jupyter Notebooks. Además, concretamente permitiría contribuir a la docencia en el curso Algoritmos y Estructuras de Datos de la FCFM.

%%%%%%%%%%%%%%%%%%%%%%%%%%%%%%%%%%%%%%%%%%%%%%%%%%%%%%%%%%%%%%%%%%%%%%%%%%%%%%%%%
