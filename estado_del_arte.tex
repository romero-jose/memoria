\chapter{Estado del Arte}

\section{Librerías de visualización de estructuras de datos}

Actualmente, existen varias librerías para generar visualizaciones de estructuras de datos, pero muchas de ellas están implementadas en lenguajes distintos de Python y no están diseñadas para poder ser utilizadas en Notebooks.
Un ejemplo de esto es la librería Reftree~\cite{Stanch2021}, que permite crear visualizaciones de estructuras de datos, pero en el lenguaje Scala, por lo tanto, no se puede utilizar para visualizar estructuras de datos implementadas en Python y tampoco se puede utilizar en Jupyter Notebooks.
Otro ejemplo es la librería Lolviz~\cite{Lolviz}, que, si bien si permite visualizar en Notebooks estructuras de datos implementadas en Python, cuenta con la limitación de que no puede generar animaciones. En la figura \ref{fig:comparacion} se pueden ver ejemplos de las visualizaciones generadas por estas librerías.

\begin{figure}[h!]%
    \centering
    \subfloat[Reftree]{
        \centering
        \includegraphics[width=\dimexpr(\linewidth-12pt)/2\relax, height=\dimexpr(\linewidth-12pt)/2\relax]{imagenes/ejemplos/reftree}
        \label{fig:reftree}
    }
    \subfloat[Lolviz]{
        \centering
        \includegraphics[width=\dimexpr(\linewidth-12pt)/2\relax]{imagenes/ejemplos/lolviz}
        % \vspace{38px}
        \label{fig:lolviz}
    }\\
    \subfloat[Aed Utilities]{
        \centering
        \includesvg[width=\dimexpr(\linewidth-12pt)/2\relax]{imagenes/ejemplos/aed}
        \label{fig:aed}
    }
    \subfloat[Manim]{
        \centering
        \includegraphics[width=\dimexpr(\linewidth-12pt)/2\relax]{imagenes/ejemplos/manim}
        \label{fig:manim}
    }
    \caption[Ejemplos de las distintas herramientas.]{Ejemplos de visualizaciones generadas por las distintas herramientas. Obtenidas de \cite{Stanch2021}, \cite{Lolviz}, \cite{aed-utilities} y \cite{manim}, respectivamente.}
    \label{fig:comparacion}
\end{figure}

Por otro lado, existe la librería Aed-Utilities~\cite{aed-utilities} creada para el curso de Algoritmos y Estructuras de Datos de la FCFM por el Profesor Ivan Sipirán; que, si bien permite generar diagramas de estructuras de datos en Python y mostrarlas en un Notebook, cuenta con las siguientes limitaciones: no permite crear animaciones, la API no es muy cómoda de utilizar y el algoritmo que utiliza para generar las visualizaciones es poco eficiente. % TODO: Explicar

Tanto Aed-Utilites como Lolviz generan los diagramas utilizando Graphviz ---una herramienta originalmente desarrollada por AT\&T para dibujar gráficos especificados en el lenguaje DOT--- que permite generar diagramas de muy buena calidad, pero no permite crear animaciones.

En cuanto a herramientas para generar animaciones, existe la librería Manim~\cite{manim}, diseñada para generar visualizaciones animadas de matemáticas. Si bien no está diseñada para crear visualizaciones de estructuras de datos, es relevante porque permite generar visualizaciones animadas y estas se pueden ver desde Jupyter Notebooks. Esta librería tiene un diseño orientado a objetos, donde una animación es un objeto que tiene un campo con el objeto matemático que representa y tiene un método que permite animar el objeto según una función de interpolación. Para generar el resultado final puede utilizar varios back-ends de renderización, incluyendo OpenGL y WebGL. Utilizando este último cuando se usa desde un Notebook. En la figura \ref{fig:manim} se puede ver una visualización de una función creada con esta herramienta.

Otra herramienta enfocada en las matemáticas es Penrose~\cite{Penrose}, una herramienta que permite crear diagramas a partir de un programa en un lenguaje específico de dominio. La disposición de los elementos en el diagrama se obtiene mediante optimización numérica. El lenguaje específico de dominio (o DSL, por sus iniciales en inglés) se separa en un archivo que define solo la parte matemática y en otro que define el estilo.

% Mostrar Penroes

% Agregar PythonTutor

Dado que no se encontró ninguna herramienta existente que permita visualizar estructuras de datos implementadas en Python, permita generar animaciones y permita mostrar las visualizaciones generadas en un Jupyter Notebooks, se requiere crear una nueva librería que implemente estas funcionalidades.

\section{Jupyter Notebooks}

Los Jupyter Notebooks son archivos interactivos que almacenan bloques de texto, código y también pueden contener imágenes, videos y animaciones interactivas. Un Jupyter Notebook tiene varios componentes que permiten construir la experiencia de usuario. En primer lugar, está el archivo del notebook, que es un archivo JSON con la extensión \texttt{.ipynb}, que almacena en el disco todos los datos persistentes del notebook, incluyendo los bloques de texto, los bloques de código y sus salidas, las imágenes, etc. Después, está el \textit{kernel} o núcleo, que es un intérprete del lenguaje respectivo, con la capacidad de comunicarse con el servidor del notebook. El servidor del notebook se encarga de la comunicación entre el front-end, el kernel y el archivo del notebook. Finalmente, está el front-end del notebook, que se encarga de mostrarle al usuario una representación del notebook y le permite a este interactuar con el notebook, usualmente corresponde a un navegador web. En la figura \ref{fig:notebook_arq} se puede ver un diagrama de esta arquitectura.

\begin{figure}[h]
  \centering
  \includegraphics[width=0.8\textwidth]{imagenes/notebook/notebook_components}
  \caption[Arquitectura de Jupyter Notebooks]{Arquitectura de Jupyter Notebooks. Obtenido de \cite{arq-notebook}.}
  \label{fig:notebook_arq}
\end{figure}

\begin{figure}[h]
    \centering
    \includegraphics[width=0.8\textwidth]{imagenes/notebook/WidgetModelView}
    \caption[Arquitectura de Jupyter Widgets]{Arquitectura de Jupyter Widgets. Obtenido de \cite{arq-widget}.}
    \label{fig:widget_arq}
\end{figure}

Existen múltiples \textit{kernels}, uno por cada lenguaje que soportan los Jupyter Notebooks, pero el más conocido es el kernel IPython, que es el kernel del lenguaje Python. Este, además de ser un kernel de Jupyter Notebooks, es un intérprete interactivo de Python, con funcionalidades como contenido multimedia y completado de comandos. Cuando se utiliza un notebook con el lenguaje Python, el servidor del notebook se comunica con este intérprete, que es el que se encarga de interpretar los bloques de código y responder con la salida generada cuando el servidor del notebook envía el request.

Para crear componentes interactivos en un Notebook de Python se utiliza la librería Jupyter Widgets. Esta librería permite definir un modelo que representa el componente, y una vista en JavaScript y HTML que es mostrada por el front-end del Notebook. El modelo se define tanto en JavaScript como en Python y la librería se encarga de mantener las dos versiones sincronizadas. Para mantener las dos versiones sincronizadas, es necesario serializar y deserializar los campos del modelo, para poder enviarlos en formato JSON. Entonces, para los campos sencillos, la librería se puede encargar de la serialización o deserialización, pero para los casos más complejos se deben definir manualmente los serializadores y deserializadores para cada campo. 

En la figura \ref{fig:widget_arq} se observa la arquitectura de un widget generado con esta librería. Aquí se puede observar como el usuario interactúa con el navegador, el cual se comunica con HTTP y WebSockets con el servidor del Notebook. Este se comunica utilizando ZeroMQ ---una librería de comunicación asíncrona orientada a mensajes--- con el \textit{kernel} y se comunica utilizando la interfaz del sistema operativo con el archivo del Notebook.

\begin{figure}[H]
    \centering
    \includegraphics[width=0.8\textwidth]{imagenes/d3/ejemplo.png}
    \caption{Ejemplo de visualización generada con D3}
    \label{fig:d3-ejemplo}
\end{figure}

\section{D3}

D3 (Data-Driven Documents) es una librería de JavaScript para manipular documentos en función de datos. Esta librería permite manipular el DOM (Document Object Model) y aplicarle transformaciones según los datos. Permite usar las tecnologías estándar de la web para generar visualizaciones de buena calidad.

Es particularmente útil la capacidad de D3 de manipular elementos SVG (Scalable Vector Graphics) ---un lenguaje de markup para describir gráficos en dos dimensiones---. Esto permite generar animaciones en dos dimensiones de forma relativamente simple. Ya que se pueden definir programáticamente los elementos SVG a partir de los datos y la herramienta permite hacer la interpolación entre el estado actual y el estado al que se quiere llegar. Además, esta técnica es bastante eficiente gracias a que los navegadores implementan motores muy eficientes para la renderización de elementos SVG.

Para generar las visualizaciones, D3 permite enlazar datos con elementos del DOM y definir los atributos de estos elementos como función de los datos. Además, cuando cambian los datos permite hacer una transición suave de los elementos correspondientes a su nueva posición según la función definida anteriormente. La figura~\ref{fig:d3-ejemplo} muestra un ejemplo de una visualización generada con D3 que representa un conjunto de datos como una agrupación de círculos. Esto muestra la versatilidad de D3 a la hora de generar visualizaciones, lo cual es una de sus principales ventajas.

\section{SUS}
\label{state-of-the-art:sus}

La escala SUS, o System Usability Scale utilizando su nombre completo, fue introducida en~\cite{brooke1996quick} por John Brooke como un método ``\textit{quick and dirty}'' (rápido y sucio) para evaluar usabilidad, desarrollado en Digital Equipment Corporation (DEC). Desde entonces, se ha vuelto un método estándar en la industria para medir la usabilidad en todo tipo de sistemas. En~\cite{evaluation-of-sus} analizan 10 años de datos de evaluaciones hechas con SUS y encuentran que esta escala es una herramienta altamente robusta y versátil para las evaluaciones de usabilidad.

SUS consiste en un formulario de 10 ítems que se le aplica a usuarios después de que utilizan el sistema que se está evaluando. Idealmente, el usuario debe contestar el formulario inmediatamente después y con sus primeras impresiones en vez de pensar mucho las respuestas. Cada ítem consiste en una afirmación para la cual el usuario debe marcar, en una escala de Likert del 1 al 5, si está muy en desacuerdo o muy de acuerdo con la afirmación. Las afirmaciones están alternadas entre afirmaciones que reflejan de forma positiva y negativa sobre la usabilidad, con el objetivo de obligar a los participantes a pensar sobre la respuesta de cada ítem. Las afirmaciones en la versión original son:
\begin{enumerate}
    \item I think that I would like to use this system frequently
    \item I found the system unnecessarily complex
    \item I thought the system was easy to use
    \item I think that I would need the support of a technical person to be able to use this system
    \item I found the various functions in this system were well integrated
    \item I thought there was too much inconsistency in this system
    \item I would imagine that most people would learn to use this system very quickly
    \item I found the system very cumbersome to use
    \item I felt very confident using the system
    \item I needed to learn a lot of things before I could get going with this system
\end{enumerate}

A partir de las respuestas del usuario se obtiene un puntaje entre 0 y 100. Para calcular el puntaje, cómo las preguntas están alternadas, primero se deben transformar los puntajes de forma que sean consistentes. Para esto, se transforma el puntaje de cada ítem a una escala entre 0 y 4, donde 0 es lo peor y 4 es lo mejor. Esto se logra restándole 1 a los ítems impares y restando de 5 los ítems pares. Después, sumando los puntajes transformados, se obtiene un puntaje total entre 0 y 40, que se multiplica por 2.5 para obtener un puntaje entre 0 y 100.

En~\cite{quantifying-the-user-experience} reportan que utilizando datos de 446 estudios y más de 5000 respuestas individuales, encontraron que el promedio del puntaje fue 68 con una desviación estándar de 12.5. Por el mismo motivo, recomiendan transformar el puntaje a un percentil para interpretar los resultados y presentan una tabla para transformar los puntajes a percentiles.

El cuestionario asociado a esta escala está en inglés, por lo que debe traducirse al español para aplicarlo a un grupo cuya lengua nativa es el español. En~\cite{spanish-sus} tradujeron el cuestionario al español, validaron el cuestionario traducido con expertos y luego lo probaron con 88 usuarios. Finalmente, concluyeron que la versión traducida del cuestionario es una herramienta válida y confiable para evaluar la usabilidad de herramientas electrónicas para usuarios hispanohablantes.
