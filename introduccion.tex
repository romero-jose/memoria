\chapter{Introducción}

Posiblemente uno de los temas más relevantes en las Ciencias de la Computación corresponde a las Estructuras de Datos. Las estructuras de datos son maneras de almacenar y organizar los datos para facilitar su acceso y modificación~\cite{Algorithms}. Comúnmente se utilizan diagramas y otras visualizaciones para facilitar la comprensión de las estructuras de datos y las operaciones que se realizan sobre ellas. En particular, las visualizaciones de estructuras de datos pueden ser útiles para la docencia en Ciencias de la Computación y pueden ser utilizadas tanto por los estudiantes como por los profesores. Las visualizaciones de algoritmos son efectivas para ayudar en el aprendizaje de estudiantes de Ciencias de la Computación~\cite{Hundhausen2002}.

Tradicionalmente, las visualizaciones de estructuras de datos que se utilizan son diagramas estáticos. Sin embargo, dado que las estructuras de datos son dinámicas y no estáticas, es útil utilizar visualizaciones que sean animaciones dinámicas de las estructuras de datos. Comúnmente las animaciones de estructuras de datos son videos o páginas web interactivas, pero en ambos casos desconectadas de donde se programan estas estructuras de datos. Por lo tanto, sería útil contar con una herramienta para generar visualizaciones animadas de estructuras de datos que se pueda usar al mismo tiempo que se implementan.

En los últimos años se ha vuelto más común la utilización de Jupyter Notebooks en la enseñanza de Ciencias de la Computación. Los Notebooks son ambientes de desarrollo interactivos basados en tecnologías web, que permiten combinar bloques de texto, bloques de código y los resultados generados por los bloques de código, para ser mostrados por una aplicación web interactiva que permite correr los bloques de código y ver los resultados generados por esto, que pueden incluir texto, imágenes, animaciones y elementos interactivos~\cite{kluyver2016jupyter}. Originalmente fueron diseñados para ser utilizados en el ámbito de la Ciencia de los Datos, pero han estado tomando popularidad en otras áreas de la computación y en la docencia.

El lenguaje de programación Python es un lenguaje de alto nivel, multiparadigma, de uso general y usualmente interpretado que se ha vuelto uno de los lenguajes más populares. Según la encuesta realizada anualmente por GitHub es el segundo lenguaje más popular, solo siendo superado por el lenguaje de programación JavaScript~\cite{encuesta-github}. Además, probablemente es uno de los lenguajes más utilizados en la docencia de Ciencias de la Computación.

Dada la popularidad de los Jupyter Notebooks y del lenguaje de programación Python, especialmente en docencia, sería beneficioso contar con una herramienta para generar visualizaciones animadas de estructuras de datos que se pueda usar en Jupyter Notebooks en el lenguaje de programación Python. De esta manera se aprovecha que los Jupyter Notebooks, a diferencia de otros ambientes de desarrollo, permite generar elementos interactivos usando las tecnologías web.

En particular, en el curso de Algoritmos y Estructuras de Datos de la Facultad de Ciencias Físicas y Matemáticas (FCFM) de la Universidad de Chile se utiliza tanto Python, como Jupyter Notebooks y se desarrolló una librería para generar visualizaciones de estructuras de datos. Esta librería tiene como objetivo de ayudar en la docencia permitiéndole tanto a los alumnos como a los profesores generar visualizaciones de estructuras de datos implementadas por ellos, por el profesor, o por algún libro o recurso externo. Sin embargo, esta librería actualmente cuenta con una serie de limitaciones: La implementación no es tan eficiente como podría ser, la API (Application Programming Interface) que se utiliza para usar la herramienta no es muy ergonómica y no permite visualizar cómo se ejecutan las operaciones sobre las estructuras de datos (solo permite visualizar una estructura de datos en un momento en el tiempo).

El problema abordado consiste en implementar una herramiento que permita generar visualizaciones animadas de las operaciones que se realizan sobre una estructura de datos implementada por el usuario. Para esto la herramiento debe ser capaz de registrar las operaciones sobre la estructura y debe ser capaz de generar una visualización animada a partir de las operaciones que registró. Es deseable que la visualización se genere en el mismo notebook y que el usuario tenga que agregar la menor cantidad de instrumentación a su código para que la herramienta funcione.

Implementar una librería que cumpla con estos requerimientos sería beneficioso para estudiantes de ciencias de la computación, profesores o personas en general, que estén aprendiendo estructuras de datos, o que deseen generar visualizaciones de estructuras de datos implementadas en Python y verlas en Jupyter Notebooks. Además, concretamente permitiría contribuir a la docencia en el curso Algoritmos y Estructuras de Datos de la FCFM.

Para lograr esto se implementó la herramienta \textit{dsvisualizer}, una librería de Python que genera visualizaciones animadas de estructuras de datos implementadas por los usuarios en Jupyter Notebooks. En su versión actual se limita a listas enlazadas, pero está diseñada para poder expandirla a otras estructuras de datos. La herrmienta está implementada como un Jupyter Widget con un backend escrito en Python que se encarga de registrar las operaciones realizadas por el usuario y un frontend escrito en JavaScript que genera las visualizaciones usando las operaciones registradas por el backend. Las dos partes de la herramiento se comunican usando la librería ipywidgets, que usa tecnologías web para mantener modelos sincronizados en el frontend y el backend.

\section{Objetivos}

\subsection*{Objetivo General}\label{sec:obj-g}

Diseñar e implementar una librería para generar visualizaciones de estructuras de datos en Notebooks de Python que sea efectiva, eficiente y usable.

\subsection*{Objetivos Específicos}\label{sec:obj-e}

\begin{enumerate}
  \item Diseñar e implementar el widget para que sea fácil de usar.
  \item Diseñar e implementar el modelo del widget.
  \item Implementar la visualización de la estructura de datos a partir del modelo del widget.
  \item Evaluar la utilidad y la usabilidad de la librería.
\end{enumerate}
