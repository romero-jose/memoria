\chapter{Introducción}

Posiblemente uno de los temas más relevantes en las Ciencias de la Computación corresponde a las Estructuras de Datos. Las estructuras de datos son maneras de almacenar y organizar los datos para facilitar su acceso y modificación~\cite{Algorithms}. Comúnmente se utilizan diagramas y otras visualizaciones para facilitar la comprensión de ellas y las operaciones que se realizan sobre ellas. En particular, las visualizaciones de estructuras de datos pueden ser útiles para la docencia en Ciencias de la Computación y pueden ser utilizadas tanto por los estudiantes como por los profesores. Las visualizaciones de algoritmos son efectivas para ayudar en el aprendizaje de estudiantes de Ciencias de la Computación~\cite{Hundhausen2002}.

Tradicionalmente estas visualizaciones han sido diagramas estáticos, pero como las estructuras de datos son dinámicas puede ser útil contar con visualizaciones animadas. Además, las visualizaciones animadas suelen estar separadas del ambiente donde se implementan las estructuras de datos y podría ser útil que estas animaciones estén integradas en el ambiente de desarrollo.

El ambiente de desarrollo interactivo Jupyter Notebooks es muy popular y es una buena opción para crear esta herramienta porque permite mostrar elementos interactivos y animados dentro del mismo ambiente de desarrollo. El lenguaje más utilizado en este ambiente es Python el cual además es el segundo lenguaje de programación más utilizado según una encuesta realizada por GitHub~\cite{encuesta-github}.

En particular, en el curso CC3001 Algoritmos y Estructuras de Datos de la Facultad de Ciencias Físicas y Matemáticas (FCFM) de la Universidad de Chile se enseñan estructuras de datos y se utiliza Python en conjunto con Jupyter Notebooks. Para este curso se desarrolló una herramienta para ayudar en la docencia que genera visualizaciones estáticas. Sin embargo, esta herramienta tiene varias limitaciones y sería útil contar con una herramienta que solucione estos problemas y permita generar visualizaciones animadas. 

Para lograr esto se implementó la herramienta \textit{dsvisualizer}, una librería de Python que genera visualizaciones animadas de estructuras de datos implementadas por los usuarios en Jupyter Notebooks. En su versión actual se limita a listas enlazadas, pero está diseñada para poder expandirla a otras estructuras de datos. La herramienta está implementada como un Jupyter Widget con un back-end escrito en Python que se encarga de registrar las operaciones realizadas por el usuario y un front-end escrito en JavaScript que genera las visualizaciones usando las operaciones registradas por el back-end. Las dos partes de la herramienta se comunican usando la librería ipywidgets, que usa tecnologías web para mantener modelos sincronizados en el front-end y el back-end.

Para validar la herramienta se realizaron pruebas con estudiantes de la FCFM. Durante estas los usuarios probaron la herramienta y luego contestaron un cuestionario. La primera parte del cuestionario es el formulario System Usability Scale (SUS) ---un método estándar para evaluar usabilidad de distintos tipos de herramientas--- y la segunda parte del formulario está compuesta por preguntas abiertas sobre la herramienta.

\section{Objetivos}

\subsection*{Objetivo General}\label{sec:obj-g}

Diseñar e implementar una librería para generar visualizaciones de estructuras de datos en Notebooks de Python que sea efectiva, eficiente y usable.

\subsection*{Objetivos Específicos}\label{sec:obj-e}

\begin{enumerate}
  \item Diseñar e implementar el widget para que sea fácil de usar.
  \item Diseñar e implementar el modelo del widget.
  \item Implementar la visualización de la estructura de datos a partir del modelo del widget.
  \item Evaluar la utilidad y la usabilidad de la librería.
\end{enumerate}
