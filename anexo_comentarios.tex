\chapter{Respuestas}
\label{anexo:comentarios}

\section{¿Qué te gustó de la herramienta? ¿Por qué?}

\begin{enumerate}
    \item Sí, fue entretenido ver como se movía cada parte de la lista y me permitió ver más fácilmente los errores
    \item Me gustó lo cómoda y fácil de usar la herramienta. No es necesario tanto conocimiento experto para usarla, y la visualización es muy intuitiva
    \item Los colores, son estéticamente agradables
    \item Que es bastante simple agregar a código que ya está escrito
    \item Logra mostrar claramente y paso a paso funciones que uno está creando. Esto ayuda enormemente a entender gráficamente lo que se está haciendo
    \item Pensando desde el punto de vista docente, me gustó porque me recordó la estructura de datos como la vi en cátedra por primera vez, las animaciones pueden ser aceleradas, lo cual me gustó. Además, el proceso que sigue la animación es bien claro y le hace completa justicia a la estructura de datos.
    \item Me gustó que fuera recorriendo el código y mostrando la visualización. Me gustó el tamaño y el color.
    \item Me gustó mucho que la visualización fuera mostrando la línea de código en la que estaba leyendo, pues ayuda enormemente a hallar errores en el código.
    \item Me gustó la idea de poder trabajar en tiempo real viendo como se construye la lista enlazada, más aún con la recursión que puede ser compleja de visualizar a primera.
    \item Me gustó que implementarlo requiera tan solo 2 líneas, es muy rápido.
    \item Ayuda a entender la construcción de una estructura paso a paso.
    \item Me gustó que fuera una visualización dinámica, es decir, muestra el proceso de los métodos llamados en la estructura de datos. Esto, porque me parece muy útil para entender lo que está sucediendo en la misma estructura.
\end{enumerate}

\section{¿Qué no te gustó de la herramienta? ¿Por qué?}

\begin{enumerate}
    \item Creo que para listas largas el tiempo en que va mostrando la lista se hace muy largo
    \item Nada, me gustó mucho
    \item No parece haber forma de quitar nodos
    \item Que no conocí de antemano las limitaciones, por lo tanto, algo que programé no servía (porque el nodo tenía dos "cabezas")
    \item No hay nada que realmente no me haya gustado. Lo único que podría mencionar es que la implementación de la herramienta se me hizo extraña, pero con la miniexplicación que se me presentó no tuve problema alguno.
    \item La visualización es buena, pero si le agrego algo de más de 6 elementos en la lista enlazada se pierde la visualización de algunas casillas. Por otro lado, si le paso listas un poco grandes como elemento a la lista, estas se representan algo más grandes que la casilla que le corresponde, aunque esto último no es realmente una molestia, sino más bien una observación.
    \item No poder ajustar la velocidad y apariencia, o que no tuviera el formato del apunte.
    \item No le encontré ningún aspecto negativo
    \item 
    \item El uso de decoradores hace que sea un poco extraño para estudiantes del curso de algoritmos, pues no los he usado antes en este curso o en cursos anteriores.
    \item Es necesario que alguien explica que indica el puntero que va apuntando hacia abajo.
    \item No me gustó que al principio no entendí por qué quedaban más nodos de los que debe haber en la lista, al eliminar nodos.
\end{enumerate}

\section{¿Cómo podría ser mejor la herramienta?}

\begin{enumerate}
    \item Sería bueno añadir un parámetro de rapidez de visualización o algo de ese estilo
    \item Quizás no poner los paréntesis en el decorador "container" y que admita múltiples objetos en la visualización.
    \item Implementando la eliminación de nodos
    \item Listas muy largas no se muestran completas
    \item Si se logra expandir y que no solo funcione para listas enlazadas.
    \item Creo que solucionar de alguna manera la visualización para listas enlazadas para hasta 10 elementos sería bueno para la herramienta, pues pensando en la docencia al poder visualizar solo un par de datos es limitado.
    \item Personalización: Color, tamaño, velocidad, fuente, figura. En su defecto, predeterminado sea parecido al apunte.

          Muestras o ejemplos: visualizar por defecto dos o más posibilidades.

          Árboles: Suele ser la parte más difícil de entender, poder jugar con esto, serviría mucho más, pues las primeras listas se ven poco (poco relevante en comparación).

    \item Expandirse a más estructuras de datos o incluir etiquetas en la visualización
    \item Solamente aumentar sus usos, entendiendo que es la primera versión, le veo mucho potencial. Sentí que me faltó ver más usos, pero que fue para ver primeras impresiones.
    \item Que haya una opción para que no muestre todo el recorrido del código a la hora de agregar un elemento a la lista, que solo se agregue al final.
    \item Limitar que la figura se pueda salir del cuadro. Al tomarlo clickeando se puede sacar de lo visible de la pantalla.
    \item Cambiando el color de los nodos que ya no forman parte de la estructura principal, o sea, los que quedan libres. Por ejemplo, en el caso de ser eliminados de la lista.
\end{enumerate}